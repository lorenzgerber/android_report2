\documentclass[a4paper,11pt,twoside]{article}
%\documentclass[a4paper,11pt,twoside,se]{article}

\usepackage{UmUStudentReport}
\usepackage{verbatim}   % Multi-line comments using \begin{comment}
\usepackage{courier}    % Nicer fonts are used. (not necessary)
\usepackage{pslatex}    % Also nicer fonts. (not necessary)
\usepackage[pdftex]{graphicx}   % allows including pdf figures
\usepackage{listings}
\usepackage{pgf-umlcd}
\usepackage{blindtext}
\usepackage{rotating}
\usepackage{enumitem}
%\usepackage{lmodern}   % Optional fonts. (not necessary)
%\usepackage{tabularx}
%\usepackage{microtype} % Provides some typographic improvements over default settings
%\usepackage{placeins}  % For aligning images with \FloatBarrier
%\usepackage{booktabs}  % For nice-looking tables
%\usepackage{titlesec}  % More granular control of sections.

% DOCUMENT INFO
% =============
\department{Department of Computing Science}
\coursename{Development of Mobile Appliations 7.5 p}
\coursecode{5DV155}
\title{User Interface for Mobile Systems}
\author{Lorenz Gerber ({\tt{dv15lgr@cs.umu.se}} {\tt{lozger03@student.umu.se}})}
\date{2017-07-20}
%\revisiondate{2016-01-18}
\instructor{Johan Eliasson / Jonathan Westin}


% DOCUMENT SETTINGS
% =================
\bibliographystyle{plain}
%\bibliographystyle{ieee}
\pagestyle{fancy}
\raggedbottom
\setcounter{secnumdepth}{2}
\setcounter{tocdepth}{2}
%\graphicspath{{images/}}   %Path for images

\usepackage{float}
\floatstyle{ruled}
\newfloat{listing}{thp}{lop}
\floatname{listing}{Listing}



% DEFINES
% =======
%\newcommand{\mycommand}{<latex code>}

% DOCUMENT
% ========
\begin{document}
\lstset{language=C}
\maketitle
\thispagestyle{empty}
\newpage
\tableofcontents
\thispagestyle{empty}
\newpage

\clearpage
\pagenumbering{arabic}

\section{Introduction}
The aim of this assignment is to translate a desktop mail client application to
a mobile app. This includes both functional and design related aspects. The
functionality shall be described in terms of Android elements and concepts such
as activities, layouts, menues, dialogs, fragments and messages. A main aspect is
to decide and reason which functionality should be stripped from the desktop version
and eventual additional functionality needed in the mobile app.

The design shall account for usability aspects following concepts from the
course litterature \cite{clark2015} and platform guidelines \cite{materialdesign}.
The report has to include several prototype designs of which at least one shall be
made in `Android Studio' and one by hand or any design/drawing application of choice.

Further, one section of the report shall describe differences and changes in the
design when the proposed Android application would be ported to another mobile
platform of choice.

\section{The Desktop Mail Client - Apple Mail}
Here the 'Apple Mail' client was chosen as desktop application to be ported to
an Android mobile app. The version at hand was 10.3 (3273) in a macOS Sierra
Environment (10.12.5). Initially, a systematic inventory of the available
functionality in Apple Mail was conducted.

\subsection{Description of main UI of Apple Mail}
The main UI of Apple Mail is shown in figure \ref{fig_apple_mail_scren}.
It consists of three columns of which only the `Mail List' and `Mail Details'
column are shown by default. The `Mail List' presents all mails of the active mailbox.
the list entry can be customised in the `Preferences', accessible through the `File'
drop down menu. The `Mail List' has by default a sort/filter bar with a drop down
menu for various list sort methods and an icon button to apply filters. The `Mail
List' scrolls vertically when not all mails of the mailbox fit on the screen.
Inspired by the Apple iOS interface, mail list items implement horizontal swipe
actions. By default, to the right for deleting and to the left for toggling
read/unread.

The `Mail Details' frame shows the detail view of one email, the one selected in
the `Mail List'. This view scrolls if needed both vertically and horizontally.
Various options regarding the visualization can be chosen in the `Preferences'
menu. By default, the header of the mail contains a number of `hyperlink' style
functionality for toggling visibility of some less often needed information but
also as shortcut for the common mail actions `Delete', `Reply', `Reply to all',
`Forward' and access to attachments.

The `Mailbox List' column can be toggled visible/invisible
by a button in the `Favorites' bar which otherwise contains text buttons for the
available mailboxes. The `Mailbox List' in combination with the `Mail List' offers
extensive `drag \& drop' functionality to put mail messages from one folder to
another.

Above the `Favorites' bar there is the `Toolbar' that contains in the default
setup nine buttons and a search field. The buttons are from left to right:
`Get new messages', `Compose new mail', `Archieve selected', `Delete selected',
`Selected to junk', `Reply', `Reply All', `Forward' and `Flag selected'. The
search field allows for text search in all or in a specific mailbox. Both
the content and the layout of the `Toolbar' is freely customizable with a number
of addtional functions/buttons not visible in the default setup.

Both the `Mail Message' and the `Folder' object on the screen provide context
sensitive menu on 'right-click'.

\begin{figure}
  \label{fig:apple_main_screen}
  \centering
    \includegraphics[width=1\textwidth]{main_screen}
    \caption{\textit{The main view of Apple Mail has three columns: `Mailbox List',
    `Mail List' and `Mail Details'. The `Mailbox List' column is however
    hidden in the standard configuration.}}
\end{figure}

\subsection{Description of Menu accessible Functionality in Apple Mail}
The `Menu Bar' contains the dropdown menus `File', `Edit', `View', `Mailbox',
`Message', `Format', `Windows' and `Help'. Most of the menu items are functionality
that is also directly accessible in the UI. The menu shows keyboard shortcuts for
much of the functionality. Menu items not found in the UI are either for
configuration and customizing the UI, or for setting up and configuring the user
data such as mailboxes accounts and smart assitant functions.


\subsection{Establishing the Mobile Application Profile}
The core functionality of a mail application is receiving, writing and sending
mail messages. Mail Message, Mail account and Mailbox administration is
secondary functionality. A desktop application like `Apple Mail' offers the
full package of primary and secondary functionality. More over, a wealth of
settings to tailor parts of the layout and application envelope according to the
user preferences.

Here it is assumed that the application profile of mobile
mail client users is by default more limited. A mobile application does not need
to offer the same flexibilty for customization and the profile of available
functions will be more narrow.

The most important functionality for a mobile mail client user is to have easy
access to the newest information. This includes receiving messages, getting
informed about new messages, quick access to new messages but also convenient
access methods for old messages. Writing new mail messages is of lower
importance. For quick informal messages most people use nowadays special
message/chat application that offer a more direct type of communication and
interaction with people. Further, it is not very convenient to type and layout
longer mail messages with the on-screen keyboard on mobile device compared to a
real physical keyboard. All sort of administration functionality besides setting
up multiple mail accounts is considered of lowest priority in the mobile mail
client.

Hence the following prioritized list of functionality resulted for the mobile
application.
\begin{enumerate}
  \item Receive and Present new Mail Messages
  \item Search for Mail Messages
  \item Write and Send Mail Messages
  \item Account Administration and Organization
\end{enumerate}



\subsection{Transforming the larger Layout Structure}
While a desktop application has very little space constraints and can
compartmentalize the main screen in different sub containers, the mobile app
uses mostly one screen for one purpose. In Android Framework terms: One `Activity'
for one purpose. For reasons of modularization and
reuseability, there is often used a second layer of abstraction, the
Fragment. As this is irrelevant for design purposes, the term `Activity' will be
used in
the whole text. The desktop app has two main containers, `Mail List' and `Mail
Details'.  In the Android app there are three main activities: `Mail List',
`Mail Details View' and `Mail Details Edit' (middle and right drawing in figure
\ref{fig:hand_design}). The desktop `Mailbox List' foldable column is translated
into a material design `navigation drawer' that slides in from the left side
(left drawing figure \ref{fig:hand_design}) \cite{navigation_drawer}.


\begin{sidewaysfigure}[ht!]
    \includegraphics[width=1\textwidth]{hand_design.png}
    \caption{Design draft on the Android Mail Client.}
    \label{fig:hand_design}
\end{sidewaysfigure}




\begin{figure}
  \label{fig:mail_message_list}
  \centering
    \includegraphics[width=0.5\textwidth]{mail_message_list}
    \caption{\textit{The main view of Apple Mail has three columns: `Mailbox List',
    `Mail List' and `Mail Details'. The `Mailbox List' column is however
    hidden in the standard configuration.}}
\end{figure}


\begin{figure}
  \label{fig:nav_drawer}
  \centering
    \includegraphics[width=0.5\textwidth]{nav_drawer}
    \caption{\textit{The main view of Apple Mail has three columns: `Mailbox List',
    `Mail List' and `Mail Details'. The `Mailbox List' column is however
    hidden in the standard configuration.}}
\end{figure}






\subsection{Interpretation and Translation of UI}
\subsection{Notifications}
\subsubsection{Mail Message Viewer}
\subsubsection{Scrollable Mail Message List}
\subsubsection{Mail Message Editor}
\section{Design with Android Elements}
\section{Some design examples from Android Studio UI Designer, some pen paper designs}
\section{describe how all main functionality in the app}
\section{Describe needed changes for another mobile platform}


\addcontentsline{toc}{section}{\refname}
\bibliography{references}

\end{document}
